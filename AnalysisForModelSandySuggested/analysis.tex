\documentclass[letterpaper,11pt]{article}

\usepackage{amsmath}
\usepackage{amsfonts}
\usepackage{graphicx}
\usepackage{natbib}
\usepackage[top=2.5cm, bottom=2.5cm, left=2.5cm, right=2.5cm]{geometry}
\usepackage{amsthm}

\newcommand{\var}{{\operatorname{var}}}
\newcommand{\cov}{{\operatorname{cov}}}
\newcommand{\cor}{{\operatorname{cor}}}
\newcommand{\E}{{\operatorname{E}}}
\newcommand{\mean}{{\operatorname{mean}}}
\newcommand{\Rp}{{\operatorname{Re}}}
\newcommand{\Ip}{{\operatorname{Im}}}

\newtheorem{theorem}{Theorem}[]
\newtheorem{corollary}[theorem]{Corollary}
\newtheorem{lemma}[theorem]{Lemma}
\newtheorem{proposition}[theorem]{Proposition}
\newtheorem{definition}[theorem]{Definition}

\begin{document}

\title{Analysis supporting the suggestion of Sandy Liebhold to improve our model, deer paper}
\author{Daniel Reuman}
\maketitle

\section{The model}\label{sec:model}

The model setup is similar to the models used in the supplementary information for 
\cite{Desharnais2018}, but in
contrast to their models, our model
considers more than two habitat patches and does not consider dispersal.
Some of our below sentences are copied with very little or no change in wording
from that reference.
Suppose population dynamics are given by
\begin{equation}
w_{i} = c_1 w_i(t-1)+ \cdots + c_n w_i(t-n)+q_0 \epsilon_{i}(t)+\cdots+q_{m}\epsilon_{i}(t-m),\label{eq:mainmodel}
\end{equation}
where $w_{i}(t)$ is the deviation of a population index $x_{i}(t)$ in
habitat patch $i$ ($i=1,\ldots,N$) at time $t$ from an equilibrium value $x_{i}^{*}$. 
The environmental noise process $\epsilon=(\epsilon_{1},\ldots,\epsilon_{N})$
is assumed to be a second-order stationary stochastic process with $\E(\epsilon_{i})=0$
for all $i$. Without loss of generality, $c_n \neq 0$ and $q_m \neq 0$. We also assume
the complex roots of the polynomial $1-c_1 z - c_2 - \cdots - c_n z^n=0$
have modulus greater than one. This assumption is a necessary and sufficient condition 
for stability of the equilibrium of the general model of which \ref{eq:mainmodel}
is a linearization [\citealp{Desharnais2018}, their supplementary information section 1.2] and also for the existence of a stationary
stochastic process that is a solution of \ref{eq:mainmodel} \citep{Desharnais2018,Brillinger2011}. 

Let $S_{\epsilon_{i}\epsilon_{i}}$
denote the power spectrum of the process $\epsilon_{i}$ and let $S_{\epsilon_{i}\epsilon_{j}}$
denote the cross spectrum of the processes $\epsilon_{i}$ and $\epsilon_{j}$
(so the cospectrum, denoted $C_{\epsilon_{i}\epsilon_{j}}$, is $\Rp(S_{\epsilon_{i}\epsilon_{j}})$, the real part). Analogously, let $S_{w_i w_i}$,
$S_{w_i w_j}$ and $C_{w_i w_j}=\Rp(S_{w_i w_j})$ be the spectrum, cross spectrum
and cospectrum of the population processes $w_i$ and $w_j$.

\section{Derivation of the main result}

Suppose we are interested in the variance (through time) of the
total population, $\var_t(\sum_i x_i(t)) = \var_t(\sum_i w_i(t))$,
and the extent to which this variance may be dominated by particular timescales.
We write $\var_t(\sum_i w_i(t))=\sum_{i,j} \cov_t(w_i(t),w_j(t))$,
and using spectral tools to decompose by timescale,
$S_{\text{tot}} = \sum_{i,j} C_{w_i w_j}$, where $S_{\text{tot}}$
is the power spectrum of $x_{\text{tot}} = \sum_i x_i(t)$. Note that
the power spectrum of $x_{\text{tot}}$ is equal to that of 
$w_{\text{tot}}=\sum_i w_i(t)$. 

\begin{proposition}
For all $i \neq j$, define the two-dimensional processes 
$w_{ij}=(w_i,w_j)$ and $\epsilon_{ij} = (\epsilon_{i},\epsilon_{j})$ and 
let $S_{w_{ij} w_{ij}}$ and $S_{\epsilon_{ij}\epsilon_{ij}}$ be the spectral
matrices of those processes. Then
$S_{w_{ij} w_{ij}} = \left| \frac{\gamma}{1-\lambda} \right|^2 S_{\epsilon_{ij}\epsilon_{ij}}$, where $\gamma=q_{ 0}+q_{ 1}\mu+\cdots+q_{m}\mu^{m}$,
$\lambda=c_{ 1 }\mu+c_{ 2}\mu^{2}+\cdots+c_{n}\mu^{n}$ and $\mu=\exp(-2\pi i f)$
for $f$ representing frequency, in cycles per time step. In particular,
$S_{w_i w_i} = \left| \frac{\gamma}{1-\lambda} \right|^2 S_{\epsilon_{i}\epsilon_{i}}$ 
and $C_{w_i w_j} = \left| \frac{\gamma}{1-\lambda} \right|^2 C_{\epsilon_{i}\epsilon_{j}}$.
\end{proposition}

\begin{proof}
Consider the two-dimensional model
\begin{equation}
w_{k}(t) =c_{ 1}w_{k}(t- 1)+\cdots+c_{n}w_{k}(t-n)+q_{  0}\epsilon_{k}(t)  +\cdots+q_{m}\epsilon_{k}(t-m)
\end{equation}
for $k=i,j$. We rewrite this as
\begin{equation}
\mathcal{C} w_{ij} = \mathcal{Q} \epsilon_{ij} \label{eq:linfiltmod}
\end{equation}
for the linear filters
\begin{align}
\mathcal{C} &= I-c_1 IB - c_2 IB^2 - \cdots - c_n IB^n \\
\mathcal{Q} &= q_0 I + q_1 IB + \cdots + q_m IB^m,
\end{align}
where $I$ is the $2 \times 2$ identity matrix. 
Here $B$ is the backshift operator,
$B^{l}\delta(t)=\delta(t-l)$, for a stochastic process $\delta(t)$.
See \cite{Brillinger2011}, \cite{Reinsel1997} and \cite{Shumway2000}
for background on linear filters. Computing the spectral matrices of both
sides of \ref{eq:linfiltmod} gives
\begin{equation}
T(\mathcal{C}) S_{w_{ij}w_{ij}} \overline{T(\mathcal{C})} = T(\mathcal{Q}) S_{\epsilon_{ij}\epsilon_{ij}}
\overline{T(\mathcal{Q})}
\end{equation}
where $T(\mathcal{C})$ and $T(\mathcal{Q})$ 
are the transfer function matrices of the linear filters
$\mathcal{C}$ and $\mathcal{Q}$, respectively (see, for instance,
section 1.1.3 of \citealp{Reinsel1997}). But 
$T(\mathcal{C}) = (1-\lambda)I$ and $T(\mathcal{Q})=\gamma I$
(section 1.1.3 of \citealp{Reinsel1997}), so
\begin{align}
S_{w_{ij}w_{ij}} &= (1-\lambda)^{-1} I \gamma I S_{\epsilon_{ij}\epsilon_{ij}} \overline{\gamma} I (1-\overline{\lambda})^{-1}I \\
&= \left| \frac{\gamma}{1-\lambda} \right|^2 S_{\epsilon_{ij}\epsilon_{ij}}.
\end{align}
$S_{w_{ij}w_{ij}}$ and $S_{\epsilon_{ij}\epsilon_{ij}}$ are $2 \times 2$ matrices, and considering the components and their real parts gives the last two statements of
the proposition.
\end{proof}

Now applying this proposition,
\begin{align}
S_{\text{tot}} &= \sum_{i,j} C_{w_i w_j} \\
&= \sum_{i,j} \left| \frac{\gamma}{1-\lambda} \right|^{ 2} C_{\epsilon_{i}\epsilon_{j}} \\
&= \left| \frac{\gamma}{1-\lambda} \right|^{ 2} \left[ \sum_i S_{\epsilon_{i}\epsilon_{i}} +\sum_{i \neq j} C_{\epsilon_{i}\epsilon_{j}}\right].
\end{align}
Thus the power spectrum of the total  population is influenced by three terms.
First, $\left| \frac{\gamma}{1-\lambda} \right|^{ 2}$, which reflects
dynamics intrinsic to the population ($\gamma$ and $\lambda$ contain
the coefficients $c_1,\ldots,c_n$ and $q_0,\ldots,q_m$). Second,
$\sum_i S_{\epsilon_{i}\epsilon_{i}}$, the  sum of the power
spectra of the noise time series, which reflects the extent to which the noise
itself may be periodic. And third, $\sum_{i \neq j} C_{\epsilon_{i}\epsilon_{j}}$,
the sum of the cospectra of the noise, which reflects the extent to which noise
synchrony across locations may be specific to certain timescales.
These three terms can be independently manipulated in the model, with some
constraints. Note that because the first sum is of $N$ quantities and the 
second is of $N^2 - N$ quantities, results 
suggest that, when $N$ is large, timescale specificity of
synchrony of noise across locations will tend to be more important than
the nature of the periodicity of the noise.

\section{Examples}



\section{Storage}

For simplicity, we make a number of assumptions of spatial symmetry. We assume
the processes $\epsilon_i$ are identically distributed  for all
$i$, so the power spectra $S_{\epsilon_{i}\epsilon_{i}}$ are all equal.
We assume
the power spectra $S_{\epsilon_{i}\epsilon_{i}}$ are nowhere zero, a reasonable
assumption because actual environmental noise typically has this broadband
characteristic. We assume
$C_{\epsilon_{i}\epsilon_{j}}=C_{\epsilon_{k}\epsilon_{l}}$ for all
$i \neq j$ and $k \neq l$; and $S_{\epsilon_{i}\epsilon_{j}}=C_{\epsilon_{i}\epsilon_{j}}$.
This last assumption amounts to assuming that the environmental noise
cross spectra are real, i.e., $\Ip(S_{\epsilon_{i}\epsilon_{j}})=0$.
This is reasonable, first, because it is consistent with spatial symmetry:
it represents an assumption
that all relationships between $\epsilon_{i}$ and $\epsilon_{j}$
are in phase, and therefore oscillations in neither variable lead to oscillations in the
other variable, at any frequency (no traveling-wave-like behavior
in the environmental noise).
The assumption $S_{\epsilon_{i}\epsilon_{j}}=C_{\epsilon_{i}\epsilon_{j}}$
is also reasonable because if we take $\epsilon_{g}$
to be an environmental process affecting all habitat patches, and we take
$\epsilon_{i}=\epsilon_{g}+\epsilon_{l_i}$, where the $\epsilon_{l_i}$
are independent local processes, then $S_{\epsilon_{i}\epsilon_{j}}$
equals the power spectrum of $\epsilon_{g}$, which is real.


Making use of the spatial symmetry assumptions, this is
\begin{equation}
S_{\text{tot}} = \left| \frac{\gamma}{1-\lambda} \right|^{ 2} \left[ N S_{\epsilon_{ 1}\epsilon_{ 1}} + (N^2-N) C_{\epsilon_{ 1}\epsilon_{ 2}} \right].\label{eq:finalres}
\end{equation}


First example will use white Gaussian noise, covariance matrix with 1s along
the diagonal and off-diagonal entries $\rho$. It will use AR(2) dynamics in each
location (stable, so without noise this has an equilibrium and it is not an
intrinsically oscillating population). Should show periodicity in the total
population.

Second example will use an AR(1) model in each location, and noise which comes
from an AR(2) model in each location where the innovations driving those models
will be standard normal, independent through time, and correlated across space.
This should end up with a peak in both the noise spectra and cospectra.

Would also like an example with peaks in the noise spectra but not in their cospectra,
but this is not strictly necessary.

Third example will use an AR(1) model in each location, and noise is white in each location, but cospectra across locations are high in a certain timescale band and low elsewhere. Not sure right now how to arrange that.

Would also like an example where noise has a peak in the spectrum, but is actually
uncorrelated across locations at that timescale band, and instead is correlated across
locations at another timescale band. Using AR(1) dynamics in each location. Should
give a counter-intuitive peak in the total-pop spectrum.

I may have to learn how to work with VARMA models to generate multi-location
noise with desired spectral structure. This seems likely to be the best way to
generate my noise the way I need it.

\bibliographystyle{plainnat}
\bibliography{biblio.bib}

\end{document}

